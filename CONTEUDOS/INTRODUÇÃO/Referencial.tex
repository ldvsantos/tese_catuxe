\pagestyle{plain}


\chapter{Referencial teórico}


\section{Desenvolvimento Rural e Sustentável: Perspectivas e Desafios no Brasil}


O conceito de desenvolvimento vai além do crescimento econômico, englobando aspectos socioculturais como saúde, educação, qualidade de vida e bem-esta (Sen, 1999). No Brasil, essa discussão ganha contornos específicos quando voltada ao desenvolvimento rural, um fenômeno que engloba não apenas crescimento econômico, mas também a melhoria da qualidade de vida, as relações sociais e a autonomia econômica das comunidades rurais \cite{Navarro2001}. 

Segundo \citeonline{Abramovay2020}, o desenvolvimento rural brasileiro segundo a teoria seniana, desenvolvimento como liberdade, ainda carece de um horizonte estratégico claro que integre ações públicas e privadas com vistas a ampliar as liberdades substantivas dos indivíduos enquanto preserva e regenera os ecossistemas dos quais a sociedade depende. Esse desenvolvimento rural sustentável envolve além das intitulações financeiras  a diversificação dos meios de vida, onde o foco não é apenas melhorar a produtividade agrícola, mas fortalecer as capacidades e resiliência das comunidades rurais frente a adversidades econômicas e ambientais \cite{Perondi2012}. 

Assim, em vez de as políticas desenvolvimentistas para o meio rural focarem apenas indicadores econômicos tradicionais, como a renda, o desenvolvimento deve contemplar também aspectos como saúde, educação, acesso à justiça e condições de vida adequadas. Segundo \citeonline{Sen1999}, o desenvolvimento autêntico ocorre quando as pessoas possuem as liberdades substantivas para escolher e conduzir suas vidas de maneira autônoma, uma perspectiva que influenciou políticas públicas e iniciativas de desenvolvimento em diversas partes do mundo. Assim, alcançar um desenvolvimento rural sustentável e duradouro requer políticas adaptadas às realidades locais, que integrem os diversos meios de subsistência e valorizem os conhecimentos específicos de cada comunidade.


\section{Papel dos Povos Originários e Comunidades Tradicionais na Sustentabilidade e Conservação Ambiental}

Neste contexto, a integração dos Povos originários e Comunidades Tradicionais no processo de desenvolvimento rural é crucial \cite{Abramovay2020}. Estas comunidades são agrupamentos sociais que mantêm uma conexão intrínseca com seus territórios, os quais são transformados em espaços coletivos significativos \cite{Kovaleva2021}. A interação desses grupos com o ambiente e com a sociedade em geral enriquece o seu saber tradicional, contribuindo para o desenvolvimento rural sustentável \cite{Montanari2020}.

O desenvolvimento rural sustentável é fundamental nesse contexto, atuando como um paradigma que alinha o crescimento econômico à preservação ambiental e à justiça social. \citeonline{Sachs2002} amplia essa visão, argumentando que o desenvolvimento rural sustentável pode ser alcançado por meio da exploração responsável dos "três Bs": biomassa, biodiversidade e biotecnologia sustentável. Esses elementos são particularmente relevantes nas comunidades rurais e tradicionais, cuja sobrevivência e crescimento dependem da capacidade de gerir de forma sustentável os recursos naturais \cite{Tagliapietra2021}.

A abordagem centrada nos três Bs permite a construção de uma civilização moderna do vegetal em países tropicais, como o Brasil, onde a interação harmoniosa entre práticas agrícolas e respeito pelo meio ambiente é essencial para o bem-estar e a resiliência das comunidades rurais \cite{Sachs1993}.

Os Povos originários e Comunidades Tradicionais são agrupamentos sociais localizados que estabelecem uma conexão profunda e duradoura com um espaço físico específico, transformando-o em um território coletivo \cite{Grant2020}. Essa transformação é fruto do trabalho dos fundadores das comunidades, que estabeleceram suas raízes nesse local. As comunidades possuem um corpo de conhecimentos e práticas singulares, derivados das interações complexas com o ambiente natural \cite{Clayton2019}. Esse saber é enriquecido tanto pela tradição quanto pela interação com dinâmicas sociais mais amplas com outros grupos sociais e com o meio ambiente.

Os conhecimentos etnoecológicos e etnopedológicos e etnoclimatológico, assim como os saberes agroecológicos das comunidades tradicionais, são aspectos fundamentais de sua relação com o meio ambiente, constituindo pilares da sua identidade cultural, ancestralidade e a sustentabilidade. Esses conhecimentos, como destacado por \citeonline{Motti2023}, englobam a compreensão detalhada dos sistemas ecológicos locais, baseada em séculos de observação e interação direta com o ambiente natural. Esta sabedoria inclui, mas não se limita a técnicas agroecológicas adaptadas, práticas de manejo de recursos naturais, identificação e uso de plantas medicinais, assim como padrões climáticos e comportamento animal \cite{Ahad2023}.

Este entendimento intrincado da natureza, que vai além do conhecimento técnico, é incorporado nas crenças, rituais e narrativas culturais das comunidades, evidenciando uma visão de mundo que harmoniza o uso e a conservação dos recursos naturais. Tais saberes, portanto, não apenas sustentam a subsistência física das comunidades tradicionais, mas também reforçam sua coesão social, resiliência e resistência diante das mudanças ambientais e sociais \cite{Brzi2023}.

Com efeito, a cultura imaterial do rural e ambiental das CPT, entrelaçada com os espaços que ocupam física e simbolicamente, é um testamento vivo de seus direitos sociais \cite{BobatoStadler2020}. Este reconhecimento de seus direitos e de seu espaço tanto no sentido físico quanto no cultural, marca um avanço significativo na valorização e proteção dessas comunidades \cite{Colao2011}.
A autonomia das comunidades tradicionais é fundamental para a manutenção da sustentabilidade de seus membros e preservação da identidade social. Essa autonomia é exercida pela troca cíclica dos Saberes Agroecológicos Tradicionais (SATs) que contribuem para a subsistência, melhorando a sua qualidade de vida, fornecendo alimentos seguros e aumentando a sua resiliência aos recursos naturais \cite{MohdSalim2023}. 

Vale destacar que na década de 1980, no contexto brasileiro, as populações residentes em Áreas Protegidas (APs) frequentemente eram subvalorizadas ou ignoradas nas políticas ambientais \cite{Marques2016}. De acordo com \citeonline{Thum2017}, desde a última década do século XX, a questão das populações tradicionais já se fazia presente nos debates sobre a diferença de modos de ser de populações. No decorrer desse processo, diferentes tentativas de definição foram colocadas em circulação. Naquele momento se formulava uma definição com caracterização capaz de incorporar as diferentes dimensões presentes nesses segmentos.

Embora, historicamente chamados planeadores ambientais considerem as comunidades locais como a principal ameaça à conservação da biodiversidade ou um retrocesso no desenvolvimento nacional \cite{Singh2000}, tal percepção negligencia o papel integral que tais comunidades desempenham na conservação da biodiversidade. Essas percepções começam a se transformar através de uma sequência de debates, embates e movimentos que culminaram no reconhecimento progressivo da importância conservacionista exercida pelas comunidades, as quais passaram a ser reconhecidas como protetoras da floresta \cite{Asante2017}.


\section{Legislação, Desafios de Implementação e Salvaguarda dos Saberes Tradicionais}


No ano de 2007, o processo de reconhecimento dos povos e comunidades tradicionais (PCTs) no Brasil atingiu um importante marco com a promulgação do Decreto nº 6.040/07 (Institui a Política Nacional de Desenvolvimento Sustentável dos Povos e Comunidades Tradicionais., 2007). Ao estabelecer um quadro legal, o decreto garante e assegura os direitos dessas comunidades ao acesso à terra, marcando um passo fundamental na consolidação de sua presença e relevância no âmbito nacional \cite{Yustiningrum2023}.

Notavelmente, o Artigo 3º, Inciso XV que dispõe ser objetivo específico da PNPCT XV – “reconhecer, proteger e promover os direitos dos povos e comunidades tradicionais sobre os seus conhecimentos, práticas e usos tradicionais” (Institui a Política Nacional de Desenvolvimento Sustentável dos Povos e Comunidades Tradicionais., 2007). Essa disposição legal é um reconhecimento de que a integração de práticas tradicionais e conhecimentos locais na gestão ambiental oferece uma perspectiva mais holística e eficaz para a conservação ambiental \cite{Ciocco2023}. Ao reconhecer formalmente estes conhecimentos e práticas, a legislação proporciona uma base para a sua proteção e promoção, assegurando que eles continuem a contribuir para a sustentabilidade ambiental e cultural \cite{Reyes-Garca2019}. 

A Lei n.º 9.985/2000, que institui o Sistema Nacional de Unidades de Conservação da Natureza (SNUC), reforça essa proteção ao estabelecer diretrizes para o uso sustentável e a preservação dos ecossistemas naturais, incluindo o reconhecimento do valor dos conhecimentos tradicionais para a conservação  \cite{Brasil2015}. Entretanto, a promulgação da Lei n.º 13.123/2015 revogou a Medida Provisória n.º 2.186-16/2001, deixando lacunas em relação à criação de um sistema formal de registro desses conhecimentos tradicionais, particularmente no que se refere ao conhecimento agrícola e outros saberes sui generis \cite{Gomes2016}.

Embora a legislação tenha avançado fortemente, ela não especifica claramente como implementar essas políticas, revelando uma lacuna na sua operacionalização \cite{Tsai2020}. Isso aponta para a necessidade de desenvolver estratégias e ações concretas que integrem e valorizem os conhecimentos e práticas das comunidades tradicionais na gestão ambiental e conservação da biodiversidade, reconhecendo sua autoria e originalidade.

Neste sentido, a proteção dos conhecimentos e saberes agroecológicos tradicionais por mecanismos de registro de conhecimento intelectual e a sua consequente transformação em reconhecimento de originalidade, além de poder solucionar distorções no que diz respeito à importância de tais expressões no plano simbólico para as comunidades tradicionais, introduzem nas mesmas uma lógica de autoria coletiva de determinada comunidade tradicional \cite{LimaVerdanRangel2018}.

Atualmente, conforme dados do Instituto Brasileiro de Geografia e Estatística \cite{IBGE2010}, o Território de identidade semiárido nordeste II abriga 18 municípios com a população total = 400 mil habitantes, estes municípios abrigam cerca de 12 comunidades quilombolas identificadas por registros administrativos e 9 agrupamentos quilombolas \cite{IBGE2022}. Essas comunidades desempenham um papel significativo em nossa memória histórica e cultural do estado (Santos; Santos, 2020). Enfatizar a relevância da Bahia nesse cenário é fundamental para promover a conscientização sobre a contribuição significativa dessas comunidades para a riqueza cultural e agroecológica do estado e do país como um todo \cite{Lima2010}.


\section{Machine Learning na Salvaguarda dos Saberes Tradicionais Agroecológicos}

O aprendizado de máquina (ML), especialmente em contextos agroecológicos, desempenha um papel crítico ao modelar interações complexas entre variáveis ambientais e socioeconômicas \cite{Zhou2018}. Em ML supervisionado, algoritmos são treinados em dados rotulados, aprendendo a prever respostas específicas (outputs) a partir de um conjunto de variáveis de entrada (inputs) conhecido. 

Para isso, a seleção de variáveis é otimizada por técnicas de redução dimensional, que eliminam redundâncias e priorizam variáveis de maior relevância para o fenômeno estudado, minimizando o risco de sobreajuste (overfitting) e mantendo a interpretabilidade. Métodos como florestas aleatórias e máquinas de vetor de suporte não apenas identificam variáveis críticas, mas também hierarquizam sua importância relativa, fornecendo insights sobre a influência de cada variável no contexto agroecológico.

Neste sentido, o uso de machine learning permite a realização de treinamentos de modelos a partir de dados empíricos, aprimorando a capacidade de prever e identificar padrões em novos conjuntos de dados  \cite{Choudhury2021}, questões estas que por vezes não são alcançadas por análises estatísticas de hipóteses. 
Em contraste aos métodos de hipóteses, os métodos de aprendizado de máquina (ML) podem ser aplicados em pesquisas guiadas pela descoberta, como abordagens indutivas ou abdutivas \cite{Ritchie2016}. Isso ocorre porque, ao contrário dos métodos tradicionais, os algoritmos de ML são capazes de identificar padrões complexos em variáveis explicativas ($X$) que se relacionam com o resultado ($y$), explorando estruturas que não foram previamente especificadas \cite{Benos2021}. 

Diferentemente dos testes de hipótese, os algoritmos de ML constroem modelos com formas funcionais flexíveis, otimizando o desempenho ao usar as variáveis explicativas ($X$) para prever o resultado ($y$) \cite{Parish2016}. Essas formas funcionais resultantes podem evidenciar relações subjacentes e inesperadas nos dados.
Em outras palavras, em vez de testar dedutivamente um modelo previamente estabelecido pelo pesquisador, como ocorre nas análises comportamentais tradicionais focada em inferência, os algoritmos de ML constroem um modelo indutivamente, com base nos próprios dados, para revelar padrões latentes \cite{Candelieri2019}. Essas características tornam o ML também uma ferramenta eficaz para a análise pós-hoc de resultados de regressão tradicionais, permitindo a identificação de padrões que poderiam passar despercebidos em uma abordagem convencional \cite{vanVliet2020}.

Assim, o presente projeto de pesquisa propõe, desenvolver um modelo empírico por meio de machine learning para salvaguarda dos saberes agroecológicos dos povos originários e comunidades tradicionais do Território de Identidade Semiárido Nordeste II na Bahia. Espera-se que o modelo se apresente como uma ferramenta que sustente a identidade cultural e ambiental do semiárido baiano, promovendo a salvaguarda dos direitos de propriedade intelectual e usufruto dos saberes e conhecimentos que respeite e reforce a singularidade desses saberes dos povos detentores.
